% template file include
% A4 lap méret, 12 pontos betűméret
\documentclass[12pt,a4paper]{article}

% szükséges csomagok
\usepackage[utf8]{inputenc}
\usepackage[T1]{fontenc}
\usepackage[magyar]{babel}
\usepackage{float}
\usepackage{graphicx}
\usepackage{pdfpages}
\usepackage{caption}
\usepackage[colorlinks=false, pdfborder={0 0 0}, linkcolor=black, unicode]{hyperref}
\usepackage{mathtools}
\usepackage{relsize}
\usepackage{amsmath}
\usepackage{amsfonts}
\usepackage{amssymb}
\usepackage{listings}
\usepackage{csquotes}
\usepackage{enumitem}
\usepackage{url}
\usepackage[nottoc,numbib]{tocbibind}
\usepackage{titlesec}
\usepackage{titletoc}

% lista elemek közti hely, és vonal mint lista elem előtti jel beállítása
\setlist{nosep, label={--}}

% tartalomjegyzék formázása
\dottedcontents{section}[1.5em]{}{1.5em}{1pc}
\dottedcontents{subsection}[3em]{}{2em}{1pc}
\dottedcontents{subsubsection}[5em]{}{3em}{1pc}

% fejezet cím formázás: 14 pontos betűméret, nagybetűs
\titleformat{\section}{\normalfont\fontsize{14}{14}\mdseries\MakeUppercase}{\thesection}{1em}{}
\titleformat{\subsection}{\normalfont\fontsize{12}{12}\bfseries}{\thesubsection}{1em}{}
\titleformat{\subsubsection}{\normalfont\fontsize{12}{12}\bfseries}{\thesubsubsection}{1em}{}

% fejezet cím térközök
\titlespacing*{\section}{0em}{0em}{1em}
\titlespacing*{\subsection}{0em}{1em}{1em}
\titlespacing*{\subsubsection}{0em}{1em}{1em}


% margó
\usepackage[right=2.50cm, 
            left=3.50cm, 
            top=2.50cm, 
            bottom=4.00cm
            ]{geometry}

% sorköz
\linespread{1.15}

% times new roman betűtípus
\usepackage{times}

% ábra és táblázat számozás beállítása fejezetenként
\setcounter{figure}{0}
\renewcommand{\thefigure}{\arabic{section}.\arabic{figure}}
\setcounter{table}{0}
\renewcommand{\thetable}{\arabic{section}.\arabic{table}}
\numberwithin{equation}{section}
\numberwithin{figure}{section}

% ábra, táblázat cím formázás
\captionsetup[figure]{labelfont={it, small},textfont={it, small},labelsep=colon}
\captionsetup[table]{labelfont={it, small},textfont={it, small},labelsep=colon}

% hivatkozások formátuma és bib fájl importálása
\usepackage[backend=biber,style=ieee,doi=false,isbn=false,url=false,eprint=false]{biblatex}
\addbibresource{references.bib}

% bekezdés behúzása
\setlength{\parindent}{5 mm}

% bekezdés térköz
\setlength{\parskip}{8pt}

% ábra és táblajegyzék formázása
\makeatletter
\renewcommand\listoffigures{
    \section{\listfigurename}
      \@mkboth{\MakeUppercase\listfigurename}%
              {\MakeUppercase\listfigurename}%
    \@starttoc{lof}%
    }

\renewcommand\listoftables{
    \section{\listtablename}
      \@mkboth{\MakeUppercase\listtablename}%
              {\MakeUppercase\listtablename}%
    \@starttoc{lot}%
    }
\makeatother




\begin{document}


\pagenumbering{gobble}

% ELŐLAPOK: borító, feladatlap, nyilatkozat, konzultációs napló
\includepdf[pages=-]{includes/BELSO-BORITOLAP_DM.pdf}
\includepdf[pages=-]{includes/FELADATLAP.pdf}
\includepdf[pages=-]{includes/HALLGATOI-NYILATKOZAT_DM.pdf}
\includepdf[pages=-]{includes/KONZULTACIOS-NAPLO_DM2_NAPPALI 1.pdf}
\includepdf[pages=-]{includes/KONZULTACIOS-NAPLO_DM3_NAPPALI 1.pdf}
\includepdf[pages=-]{includes/KONZULTACIOS-NAPLO_DM4_NAPPALI 1.pdf}



\section*{ABSZTRAKT}
Absztrakt magyarul.

\newpage
\section*{ABSTRACT}
Abstract in english.



% TARTALOMJEGYZÉK
\newpage
\pagenumbering{arabic}
\tableofcontents
\newpage


% -----------------------------------------------------

% ITT KELL HOZZÁADNI A FEJEZETEKET, CÍM NAGYBETŰVEL:

\clearpage
\section{BEVEZETÉS}
\input{chapters/1_Bevezetes}

\clearpage
\section{PÉLDA FEJEZET}
\subsection{Ábrák}
	
Ez itt a \ref{fig:ur5} ábra.
\begin{figure}[H]
	\centering
	\includegraphics[width=0.5\linewidth]{img/ur5robot.png}
	\caption{UR5 kollaboratív robot}
	\label{fig:ur5}
\end{figure}


\subsection{Hivatkozások}

Ez itt egy folyóirat cikk \cite{journal-example}.

Ez itt egy konferencia cikk \cite{conference-example}.

Ez itt egy könyv \cite{book-example}.

Ez itt egy online forrás \cite{online-example}.

Ez itt egy disszertáció \cite{thesis-example}.

	
\subsection{Táblázatok}
Példaként itt látható egy táblázat \ref{tab:ur5}.

\begin{table}[H]
	\centering
	\begin{tabular}{|c|c|}
	    \hline
		Tulajdonság                      & Érték    \\ \hline
		kinyúlás {[}mm{]}                & 850      \\ \hline
		Szabadságfok                     & 6        \\ \hline
		Teherbírás {[}kg{]}               & 5        \\ \hline
		Súly {[}kg{]}                    & 18,4     \\ \hline
		Ismétlési pontosság {[}mm{]}     & $\pm0,1$    \\ \hline
		Teljesítményfelvétel {[}W{]}     & 90-325   \\ \hline
		Csuklók mozgástartománya {[}$^{\circ}${]} & $\pm360$    \\ \hline
		Max. csuklósebesség {[}$^{\circ}/sec${]}  & $\pm180$    \\ \hline
		Max. Tool sebesség {[}m/s{]}     & 1        \\ \hline
		Programozási nyelv               & URscript \\ \hline
	\end{tabular}
	\caption{Az UR5 robot főbb paraméterei}
	\label{tab:ur5}
\end{table}

Egy másik a \ref{tab:joint-limits} táblázat.

\begin{table}[H]
    \centering
    \begin{tabular}{|c|c|c|c|c|}
        \hline
    	Robotcsukló & \begin{tabular}[c]{@{}c@{}}min. poz.\\ {[}rad{]}\end{tabular} & \begin{tabular}[c]{@{}c@{}}max. poz.\\ {[}rad{]}\end{tabular} & \begin{tabular}[c]{@{}c@{}}max. sebesség\\ {[}rad/s{]}\end{tabular} & \begin{tabular}[c]{@{}c@{}}max. gyorsulás\\ \hline {[}rad/s\textasciicircum{}2{]}\end{tabular} \\ \hline
    	1 & -0.802 & 1.39 & 3.1 & 9 \\ \hline
    	2 & -2.25 & -0.873 & 3.1 & 9 \\ \hline
    	3 & 1.25 & 2.7 & 3.1 & 9 \\ \hline
    	4 & -3.49 & -1.41 & 3.1 & 9 \\ \hline
    	5 & -2.62 & -0.62 & 3.1 & 9 \\ \hline
    	6 & -3.14 & 2.09 & 3.1 & 9 \\ \hline
    \end{tabular}
    \caption{Másik példa táblázat}
    \label{tab:joint-limits}
\end{table}

	
\subsection{Képlet}

Ez itt a \ref{eq:keplet1} képlet:
\begin{align}
    \cos{\alpha-\beta} = \cos{\alpha}\cdot\sin{\beta}-\cos{\beta}\cdot\sin{\alpha}
    \label{eq:keplet1}
\end{align}	

	
\subsection{Felsorolás}

Ez egy felsorolás:
\begin{itemize}
	\item A felsorolás első eleme
	\item A felsorolás második eleme
	\item A felsorolás harmadik eleme
\end{itemize}










% ------------------------------------------------------

% JEGYZÉKEK, INNEN KI LEHET KOMMENTEZNI AMELYIK ESETLEG NEM KELL (de ha van ábra/táblázat akkor kötelező, irodalomjegyzék kötelező, rövidítések opcionális)

\clearpage
\section{IRODALOMJEGYZÉK}
\printbibliography[heading=none]

\clearpage
\renewcommand{\listfigurename}{ÁBRAJEGYZÉK}
\listoffigures

\clearpage
\renewcommand{\listtablename}{TÁBLAJEGYZÉK}
\listoftables

\clearpage
\section{RÖVIDÍTÉSEK}
\newcommand{\acronym}[2]{
    \item [\textbf{#1}] #2
}

\begin{itemize}[labelwidth=3cm,align=left,itemindent=3cm,itemsep=4pt]

% IDE LEHET HOZZÁADNI A RÖVIDÍTÉSEKET:

    \acronym{UR}{Universal Robots}
    \acronym{ROS}{Robot Operating System}
    \acronym{NASA}{National Aeronautics and Space Administration}

\end{itemize}





\end{document}